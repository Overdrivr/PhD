
\subsection{Current limitations and conclusion}

%TODO: Make section ? Speak about limitations with multiples nets that are responsible for error propagation
%TODO: Limitation with multiple nets. Interactions between inputs.
%TODO Review next
% Where does error come from ? Load impedance
Previously, it was indicated that all WnB curves were extracted with a fixed Zout = 1M\textOmega.
When connected together, each block (pre-regulator, bandgap, regulator) sees a load impedance on its output much different than that.
Thus, it is important to evaluate the impact of this value on the Wunsch and Bell characterization curve.
The pre-regulator is characterized again, this time with a load impedance of 100\textOmega.
This value is rather small impedance, but sufficiently high to not draw too much current on the pre-regulator.
This second characterization is given fig. X.

%TODO: CHARACTERIZATION WITH Z LOW IMPEDANCE

This characterization is compared with the one extracted with 1M\textOmega\.
To make the comparison easier, curves for failure above 1ns, 10ns, 100 ns and 1us were plotted separately (see fig. X).
MAKE COMPARISON

%TODO: COMPARISON Z LOW IMPEDANCE Z HIGH IMPEDANCE 2 * 2 1n -> 1u

SO FAR, CONSIDERED IMPEDANCE AS STATIC AND REAL.
TALK ABOUT IMAGINARY IMPEDANCES, AND DYNAMIC IMPEDANCES

TALK ABOUT ERROR CAUSED BY GRADIENT ?




On paper, this method was rather promising in terms of applicability.
A block could be characterized once, and reused in different places.
The robustness of a full system could have been quickly and easily deduced from the models of its parts.

However, with the study case exposed earlier, several issues arose that clearly limit the ability of the model to perform as expected.

The main issue with this modelling method is the fact that it relies too much
on the value of the output load for performing the characterization of a block.
This load depends on many different parameters.
And this value will change in function out the block connected on the output (think about block-reuse)
Also, this value may not be constant in frequency.
And this value may also not be constant in time (multiple operating modes, biasing points, etc).

% SHOW SIMULATIONS FOR THIS PROBLEM OF IMPEDANCE VARIATION

% DIRECTIVITY - WE ASSUME STRESS AND FAILURES PROPAGATE FROM INPUT TO OUTPUT. MAY NOT BE THE CASE. ALSO, MAY NOT WORK IN REVERSE WITH SINGLE2MANY BLOCK CONNECTIONS

Another issue, this method is limited to a binary FAIL/NO FAIL criteria.
Not only this criteria is arbitrary (in some cases, the specification could be used to set it), but
for most purely-analog blocks, there will not be a clear failure, rather, most
nets will have degraded values until maybe biasing of the block completely fails.
In this case, the binary criteria hides a lot of information about the degradation.

ANY CLUES TO OVERCOME THESE ISSUES ?

ALSO, FAILURE MAY NOT PROPAGATE IN THIS ONLY WAY, BUT BACKWARD TOO

The main reason why this approach was investigated was for its very interesting modularity
that was highly suitable for block reuse workflow.

However, because of the intrisic interactions between block functions in an integrated circuit,
TRY TO EXPRESS BETTER WHAT PRINCIPLE OF THIS METHOD BOUNDS IT TO FAIL

this approach seems to be bound to fail for building a reusable model for an IC block that could predict
ESD failures at the architecture level and during the IC architecture phase.
