\section{Black box modeling approach}

Unlike the bottom-up modeling approach, the black box method studies the relationship of
between Input/Outputs of the complete integrated circuit.

This approach abstracts the IC internal complexity with a single model focused on
describing the failure of an output when an input is stressed.

The main advantage compared to the bottom-up method, is that where there was no
way to set a clear failure criteria with the bottom-up, using the black box method, the
product specification serves as a clear reference for defining failures.

Indeed, we can consider using this specification to say that, during an ESD,
any signal that goes outside its nominal range is at fault.

Then, the idea is to inject the ESD on an input pin, and observe if some output
pins go out of specification, and so, in fault.

Another advantage of this method is the abstraction over the design. Since it is
easy to probe pins of the IC, we can measure voltage/current at those pins fairly
easily, and derive models from those values, without needing the access the chip's internals.

Overall, the aim of the black-box model is make abstractions of internal complexities, which is very suitable in our case.

\subsection{Characterization}
To build the black-box model that links an output with an input, a characterisation is needed.
Once again, a WB method is used. A variable width/variable amplitude stress is applied on the input pin.
The output pin is then observed to detect at which (width, amplitude) it is going out of specification.

This characterization is performed on the testchip, still on the pre-regulator/bandgap/regulator supply chain.
The stress is injected on the pre-regulator VBAT pin (input of our black-box).
The output is the output of the regulator V2P5, which is supposed to deliver a 2.5V regulated supply.

Since the injection is performed on a supply, an injection method is required to superimpose a stress on the DC voltage,
and to make sure that the stress will not propagate inside the supply but preferably inside the IC.

A very popular method to superimpose a stress on a DC voltage is using a bias tee as described in the DPI standard (REF ?).
This standard targets EMC disturbances injection (frequency domain),
but was employed at multiple times (REFS ?) for time-domain disturbances injection.

The bias tee schematic is given Fig. X.

BIAS TEE SCHEMATIC

This injection method was initially used for performing the powered-on characterization.

Fig X shows the waveform of the VBAT injected current when stressed with a TLP (square) impulse.
The TLP waveform before the capacitor is shown for reference.

WAVEFORM CURRENT VBAT AND TLP BEFORE INJECTION CAPACITOR

Obviously, there is an issue with the injection setup ? Current not square ?

...

Finally, the characterization curve is obtained.
On x-axis, there is the pulse width.
On y-axis, there is the minimal stress amplitude that caused a failure on the output (V2P5 < 2.3 V ?)

Now that the (VBAT,V2P5) response has been characterized, a model of the failure can be constructed.

\subsection{Failure modeling}

To model the failure, a verilog-ams model is written.
It continuously checks the input current and deduces if the output is going out of spec.

Algorithm:
- Take minimum failing threshold input current
- If on the input, current goes above this threshold
  - start counting time
  - If time goes above corresponding failure pulse width
    - mark this level as failure
  - If on the input, current goes above the next threshold
    - start counting time
    - If time goes above corresponding failure pulse width
      - mark this level as failure
    - Etc.

At the end, list of all fails.

TODO: Improve algorithm

\subsection{Input modeling}

Along with the black-box failure model, a black box input model is required to run a SPICe simulation.

TLP characterization ?

\subsection{Validation on a ESD Gun stress}
Use it in other simulations (Gun for instance), to check if failure level can be predicted
