\section{Top-down modeling approach}

Unlike the bottom-up modeling approach, the top-down considers an IC in terms of
externally-exposed IO, and a specification regarding the state of these IOs.

The main advantage compared to the bottom-up method, is that where there was no
way to set a clear failure criteria with the bottom-up, using the top-down, the
product specification serves as a clear reference for defining failures.

Indeed, we can consider using this specification to say that, during an ESD,
any signal that goes outside its nominal range is at fault.

Then, the idea is to inject the ESD on an input pin, and observe if some output
pins go out of specification, and so, in fault.

Another advantage of this method is the abstraction over the design. Since it is
easy to probe pins of the IC, we can measure voltage/current at those pins fairly
easily, and derive models from those values, without needing the access the chip's internals.

The question is, is this sufficient to build an accurate enough model ?

\subsection{Characterization}
Apply variable width/variable height stress

Talk about injection setup. Move away from DPI system

\subsection{Failure modeling}
Make failure model + IO model
Use it in other simulations (Gun for instance), to check if failure level can be predicted
