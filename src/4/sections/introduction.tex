% Electronic schematics are composed of transistor models
Conception flow of integrated circuits is based on electronic schematics that describe the circuit using symbolic view
This view is then converted into a netlist composed of all components inside the circuit and connections between them.
At silicon-level, the components are mostly capacitors, resistors, and transistor devices.
Those components are all described by their individual model.
Those models are a mathematical or algorithmic representation of the behavior of a device.
In electronics, it links current and voltages between the different terminals of the device.

% Circuits are complex which raises all kinds of problems
In the automotive field, ASICs used in ECUs are very complex.
They must communicate with external sensors, process and interpret the data, then apply commands on other external devices.
It is a lot of functionnality to pack into the integrated circuit.
In return, schematics are very complex and dense.
Simulating ESD injection into the circuit is a top cell simulation where the simulator must solve voltage and current in all circuit nodes, while dealing with very fast and large amplitude ESD signals.
It raises big convergence issues, that can prevent the simulation to complete.
When the simulation does complete, simulation time can be very important, preventing fast feedback, iteration, and investigation.
All those issues increase a lot the investigation time, because ultimately the goal is to debug the circuit functionnality when exposed to electrical fast transients.

% What is not taken into account in those transistor level models
Also, those models and schematic are not perfect and do not take everything into account.
Parasitic devices can be extracted from the layout, but require a layout which often is not designed until the end of the conception flow.
Also, extraction is a time consuming and complex task, that takes quite some time to be computed.
Finally, some phenomena such as substrate coupling are simply not reproduced and need special tooling for simulating them.

% Talk about SEED and why it is a main driver for this chapter
SEED methodology
Big trend in the ESD field (Industry council)
Recommends to size ESD protection with a collaboration between on-chip and system-level.
Result of the shift of responsabilities on holding system-level specs at silicon level.
It is not smart from a cost point of view.
Goal of SEED is to provide the most efficient and cost-effective solution.
A combination of protecting at system level and at silicon level.
SEED applies so far to hard-failure.
Ultimately, this trend will apply to soft-failure as well once the topic starts to be more mastered by the community.
Problem nowadays is that there is no models of analog IC core functions that can be freely distributed.
Distributing the transistor-level schematic is not conceivable.
Reveal the conception detail

% Black box model is a solution
Black box models are a nice solution to the issues and limitations described previously.
They only reproduce the behavior of the device from an external point of view, without knowing intrinsic details.
Because they abstract all the inner complexity, and are much simpler, they help drastically reducing simulation time of transistor-level schematics.
The also hide the inner functionnality and conception details.
They can be distributed freely without revealing intellectual property.

%TODO: Talk about design flow
%TODO: Talk about bottom-up top down

% What is done in this chapter
In this chapter, three different kinds of modelling approaches at the integrated circuit level are explored.
A bottom-up modelling method is presented first.
It focuses on modelling mathematically interaction between an input and an output of invidual block functions inside the chip.
Those individual blocks can then be chained together to deduce the behavior of the complete functionnality against a transient disturbance, using a much simpler model than the transistor-level schematic.
A similar method is applied to a complete IC function.
The goal is a bit different because the goal is to simulate electrically an IC function with a black box model.
A characterization approach is presented, and current limitations are highlighted.
