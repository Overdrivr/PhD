% Electronic schematics are composed of transistor models
The conception flow of integrated circuits is based on circuit schematic, a form of symbolic view describing an electronic circuit.
Those schematics are converted into a netlist composed of all components inside the schematic and connections between them.
At silicon-level, these components are mostly capacitors, resistors, and transistor devices.
Each element is described by a model, which is a mathematical or algorithmic representation of their behavior.
In electronics, models usually link current and voltages between the different terminals of the device, and can accept external parameters such as temperature.
With the netlist and the model library, the simulator knows the stimulii, the devices and the connections inside the circuit.
It is able to simulate circuit behavior, allowing for validation of the functionnality very early in the design phase.
As detailled in the global introduction, this is the main reason why simulations are so valuable and why simulation tools for ESD can avoid late and costly mistakes.

% Circuits are complex hierarchies
In the automotive field, \gls{asic}s used in electronic modules are very complex.
They must communicate with external sensors, process and interpret the data, then apply commands on other external devices.
It is a lot of functionnality to pack into the integrated circuit.
In return, schematics are very complex and dense.
To organize the design, a hierarchy is set in place, similar to a tree.
The top-level cell has child cells themselves having child cells, until root cells composed of transistors, passive devices, etc.
The design phase usually starts from the bottom, where basic blocks are designed with library devices, then blocks are connected together in rising complexity up to the top.
For that reason, the design phase is usually called bottom-up.
After the design phase, all the schematic views are converted and drawn into a layout view.
The layout view represents the set of masks and layers that will be used during manufacturing to build the circuit.

% Complex simulations too
Simulating the injection of an ESD into the circuit is rather challenging, because the simulator must solve voltage and current in all circuit nodes, while dealing with very fast and large amplitude ESD signals.
It raises big convergence issues, that can prevent the simulation to complete.
When the simulation does complete, simulation time can be very important.
To successfully debug a circuit, it takes multiples iterations and design tweaks to reach the desired goal.
Therefore, the investigation time can be rather large.

% What is not taken into account in those transistor level models
Also, those models and schematic are not perfect and do not take everything into account.
Parasitic devices can be extracted from the layout, but require a layout which often is not designed until the end of the conception flow.
Also, extraction is a time consuming and complex task, that takes quite some time to be computed.
Finally, some phenomena such as substrate coupling are simply not reproduced and need special tooling for simulating them.

% Talk about SEED and why it is a main driver for this chapter
The \gls{seed} methodology is a novel approach for efficiently designing ESD robust applications.
It is a significant trend in the ESD field supported by the Industry Council \cite{seed}.
This methodology recommends a collaboration the chip and the system to handle the \gls{esd} robustness.
The goal of this method is to provide the most efficient and cost-effective solution for designing products, with a combination of protections at the system level and at silicon level.
So far, \gls{seed} applies to hard-failure.
It is believed that this trend will also apply to soft-failure once the topic starts to be more experienced by the community.
The major lock nowadays is the lack analog IC function models that can be freely distributed, and that would allow equipment manufacturers to perform complete ESD simulations with the integrated circuit model.

% Black box model is a solution
Black box models are a potential solution to the issues and limitations described previously.
They only reproduce the behavior of the device from an external point of view, without knowing intrinsic details.
Because they abstract all the inner complexity, and are much simpler, they help drastically reducing simulation time of transistor-level schematics.
The also hide the inner functionnality and conception details.
They can be distributed freely without revealing intellectual property.

% What is done in this chapter
In this chapter, three different kinds of modelling approaches at the integrated circuit level are explored.
A bottom-up modelling method is presented first.
It focuses on modelling mathematically interaction between an input and an output of invidual block functions inside the chip.
Those individual blocks can then be chained together to deduce the behavior of the complete functionnality against a transient disturbance, using a much simpler model than the transistor-level schematic.
A similar method is applied to a complete IC function.
The goal is a bit different because the goal is to simulate electrically an IC function with a black box model.
A characterization approach is presented, and current limitations are highlighted.
