\section{Hierarchical bottom-up assertions}

Principle of an assertion. Assertions are written for each block (each cell in Cadence).

How to write assertions ?
Use specification of the block + a few rules of thumbs (for instance, ground nets should not be above, below 1V).

Why are assertions useful ?
Inherent scale factor and complexity makes inspecting the top-cell and all its sub cells impossible for a human.
Need an automated checking system, that runs for any simulation (ESD or other).

Applicability?
Determine automatically in a top simulation (or simplified top) which blocks/nets went out of spec and so at fault.
Also, determine mistakes when building testbenches.
Also, constitute an electrical documentation of the block block.
Also, determine during DC spec if all blocks are properly started.
Also, determine connection issues (floating ground and nets).

Potential integration in Cadence environment
So far, limited support for assertions in Cadence.
In an ideal case, assertion files should be a specific view of the asserted cell.
This way, when the cell is copied or moved around, the assertions will move with it and be reused.

Major perks
Simple to write.
Reusable.
Integrates well into the design flow
General purpose (not limited to ESD, very useful in general considering all applications).

User interface proposal for monitoring assertions.
Visualize when circuit is ready (all assertions are gree)
During and after ESD, which nets/blocks are disturbed first, how disturbances propagate inside the circuit.
