\section{Principle of operation of the TLP-HMM}

composed of two modules that plug directly onto a standard TLP.
These modules are simply referred hereafter as “absorber” and “shaping filter” (Fig. 3).
The principle is to re-route a part of an incoming TLP pulse into the ground.
The remaining current injected on a 2ohm calibration resistor (same as ESD gun cal) claims the same waveform.

%TODO: SCHEMA

\subsection{Role of the shaping filter}

Role of the shaping filter ?
Details of the shaping filter
a capacitor C is separated from the main propagation path by a small transmission line (t).
The inductor L can be neglected in this first part of the analysis, as it behaves as an open circuit at the beginning of the pulse.
The short line introduces a delay t between the main line and the RLC elements.
When a TLP pulse is injected on the main line, it reaches point A at t=0 and the voltage at A rises from 0V.
The capacitor is still “not visible” from A and does not see the TLP rising edge yet.
At t=t, the pulse has reached the capacitor which is initially discharged and starts charging.
This change in voltage and current in the capacitor branch at t=t is not visible immediately from point A until it propagates back.
Point A keeps rising with the TLP impulse.
At t = 2t, the reflection from the capacitor reaches A, whose potential falls almost immediately.
This results in the generation of the short first section.
The peak width of approximately 2t on the pulse going to the load.
The last element to be explained is the resistor R.
It introduces an offset voltage as soon as the capacitor C starts charging.
This offset is used to tune the voltage and (under 1  the current I(t) upwards or downwards at t =2t.
After 2t, point A follows the charging capacitor voltage.
If the resistor R and inductor L were not connected along with the capacitor, the charge would continue until TLP voltage is reached (with reflections from the short cable neglected).
This is where the inductor L connected in parallel plays a part.
At t = 0, the inductor is an open circuit and conducts approximately no current.
Slowly the current through the inductor increases, and at t= t1, it is enough to cancel the capacitor charging current.
At this moment, the capacitor voltage increases has stopped and the capacitor starts to discharge through the inductor.
Ultimately, the inductor draws all the current and brings the voltage and current on the main line back to 0.
The result of this combined action after 2t of the capacitor and the inductor leads to the generation of the second part of the pulse.

\subsection{Role of the absorber}

Role of the absorber

\subsection{Shaping filter design}

The exact schematic of the shaping filter is given Fig. .
Capacitances are distributed to reduce parasitic inductances.
The inductances have also been distributed to increase the maximum total current that can be absorbed.
The PCB (Fig. ) has a ground plane, and the central line is 50 matched.
Overall, its dimensions must be kept as small as possible to reduce the impact of delays.

%TODO: Fig schematic
%TODO: Picture

In Figure , the delay cable (t) can be seen on the left.
As described in II.1, this cable must induce a delay half of the final pulse risetime.
To be compliant, the rise time of the pulse must be comprised between 700ps and 1ns.
Thus, the delay t must be comprised between 350ps and 500ps to have a clean peak.
In practice, a slightly longer cable ensures the maximum (and desired) peak voltage is reached.
The pulse rise time is fixed directly by the TLP rise time, which can be enforced accurately with a rise time filter [9][10].

%TODO: Figure impact delay

Compared to the TLP risetime, a shorter delay for the shaping filter’s cable will reduce the amplitude of the peak while a longer delay will let the TLP reach its maximum voltage and generate a short flat region. The impact of different t values on the peak is presented on Fig 7. The TLP charging voltage is identical for each curve and the load is 2 calibration load defined in IEC 61000-4-2). A 2ns rise time filter is employed here for accentuating the interaction between the risetime of the pulse and the delay of the cable. In the final system, a 1ns risetime filter is used instead to comply with the HMM specification.
The red curve in Fig. 7 corresponds to the shortest cable. The peak amplitude never reaches the maximum voltage. The blue curve corresponds to the optimal length. The peak reaches the maximum voltage and falls immediately. The green curve corresponds to the longest cable. The delay is long enough that the load sees a TLP step for 10ns before the voltage falls down.
The last element playing a part in the pulse shaping is the series resistor RS (Fig. 3) of value 8 . With this resistor, it is easier to match the required ratio between the peak current and the 30 ns current.

\subsection{Absorber design}

The schematic of the absorber is given Fig. 8.
It is constituted of a 50 resistor, in series with 6.6 nF.
It acts as a matched termination for transient events, and absorbs any incoming reflections.

%TODO: Schematic of absorber

Because it is connected to the TLP line but on the opposite side of the load, it will absorb current at the end of the TLP pulse.
Thus, the pulse reflection of opposite sign will be globally eliminated by this system.
The picture of the absorber is given Fig. 9.
It is helpful for understanding a small issue caused by a parasitic capacitance described in Section II.5.

%TODO: Picture
