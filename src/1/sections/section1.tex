\section{Context}
\subsection{Electrostatic discharge}

%TODO
Electrostatic discharges are the result of the accumulation of potential.
Caused by tribocharging or electrostatic induction.
Sufficient potential is reached, a discharge happens.

% What are its key properties
Electrostatic discharges are very short events, usually in the range of a few hundred nanoseconds.
Currents in the range of tens of amperes are frequently observed and voltages of thousands of volts.
As a result, the overall discharge energy is small, but the power is extremely high.

% Impact
Harsh electrical disturbances are very harmful for electronic systems and integrated circuits.
Today, there is not a single electronic device in the field that is not protected against electrostatic discharges.
Due to the very short, fast, and random nature of \gls{esd} events, it remains very challenging to perfectly protect an electronic system against them.

% Especially for ESD
The automotive field has also the harshest requirements for \gls{esd}.
Several mechanisms generate static electricity accumulation when a vehicle is in motion \cite{generationESDautomotive}.
When electrostatic potentials become large enough, dielectric barrier breaks down and a discharge happens.
It propagates through the car's body, wiring and electronic equipments.
Electronic devices inside a car are therefore constantly exposed to electrostatic discharges.
They must be protected against it to avoid hardware failures or operating issues.

% Triboelectrification between tires and road surface
Triboelectrification can occur between the tires and and the road surface when a car is in motion \cite{generationESDautomotive}.
This process leads to the accumulation of electrostatic potential inside the car.
The capacitance formed between the vehicle and the ground effectively charges.
This effect is potentially countered by the vehicle leakage current to the ground.
If the triboelectric current is larger than the leakage, the capacitor charges anyway, until sufficiently a high potential potential is reached, at which point an electrostatic discharge is likely to happen.

% Triboelectrification between body and flow of air
%TODO: Find reference for this phenomenon
Troboelectrification can also occur for between the vehicle's body and the airflow generated by the motion.
More specifically, it is not the air itself but the particles carried in it that induce
Similarly, the friction between the car's body and the air flowing through it generates charge accumulation on the vehicle.

% Triboelectrification between a human body with clothing and the seat fabric
Triboelectrification may also happen between a human body with clothing and the seat fabric.
The probability of discharge is rather low \cite{generationESDautomotive} and may mostly happen when the human body leaves the vehicle, but it is still a possible source of \gls{esd}.
