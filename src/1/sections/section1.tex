\section{Context}
\subsection{Electrostatic discharge}

% What is an ESD, its key properties
An electrostatic discharge (ESD) is the result of the accumulation of electrostatic potential.
It is a very sudden current flow that propagates through metallic or non-metallic materials, electrical systems and sometimes even through the air.
It is a very short electrical event, involving large currents and extremely high voltages.
\gls{ESD} have durations in the range of a few hundred nanoseconds.
Currents can reach tens of amperes and voltage levels several thousand of volts.
The total discharge energy is small, somewhere near the milli-Joule (mJ).
The power on the other hand is extremely high because the waveforms are changing extremely rapidly.

% What generates an ESD
Objects can accumulate electrostatic potential by tribocharging or electrostatic induction.
In the case of tribocharging, electrons are transferred from one object the other when they are put into contact then separated.
One object becomes positively charged and the other negatively.
Which object gains electrons depends on the kind of material constituting each object.
Nature of common materials in that regard is given in Table \ref{tab:chargin-sign-table}

\begin{table}[!h]
\centering
\begin{tabular}{@{}ll@{}}
\toprule
↑ Positive & Rabbit fur     \\
           & Glass          \\
           & Mica           \\
           & Human Hair     \\
           & Nylon          \\
           & Wool           \\
           & Fur            \\
           & Lead           \\
           & Silk           \\
           & Aluminum       \\ \midrule
           & Paper          \\
           & Cotton         \\
           & Steel          \\
           & Wood           \\
           & Amber          \\
           & Sealing Wax    \\
           & Nickel, copper brass, silver \\
           & Gold, platinum \\
           & Sulfur         \\
           & Acetate rayon  \\
           & Polyester      \\
           & Celluloid      \\
           & Silicon        \\
↓ Negative & Teflon         \\
\bottomrule
\end{tabular}
\caption{Typical Triboelectric Series (Credit: ESDA \cite{esda-triboseries})}
\label{tab:chargin-sign-table}
\end{table}

% Field induction
Field induction, also called electrostatic induction, works differently but also results in accumulation of electrostatic potential.
When a piece of material is placed inside an electric field, positive and negative charges of the material become spatially separated, at a macroscopic scale.
If the object is temporarily grounded in one point, charges near that location are evacuated, in majority positive or negative.
Similarly to tribocharging, electrostatic induction causes the material to loose its electrical neutrality.

% How are they generated in the automotive field - Triboelectrification between tires and road surface
Tribocharging happens constantly when a vehicle is in motion \cite{generationESDautomotive}.
It occurs because of the friction of the tires on the road surface, which is a constant contact followed by separation mechanism.
Therefore, static electricity accumulates constantly inside a vehicle, until a discharge happens.

% Triboelectrification between a human body with clothing and the seat fabric
Triboelectrification may also happen between a human body with clothing and the seat fabric, eventually leading to an electrostatic discharge.
The probability of discharge is rather low \cite{generationESDautomotive} and the phenomenon can be only be observed in rare situations when the human body leaves the vehicle.

% Triboelectrification between body and flow of air
%TODO: Find reference for this phenomenon
Similarly, triboelectrification can also occur between the vehicle's body and the airflow generated by the motion.
More specifically, it is not the air itself but the particles carried in it that induce the contact and separation process.
