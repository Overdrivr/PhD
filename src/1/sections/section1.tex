\section{Context}
\subsection{Electrostatic discharge}

% What is an ESD, its key properties
An electrostatic discharge (ESD) is the result of the accumulation of electrostatic potential.
It is a very sudden current flow that propagates through metallic or non-metallic materials, electrical systems and sometimes even through the air.
They are very short electrical events, involving large currents and extremely high voltages.
They usually last a few hundred nanoseconds, with currents in the range of tens of amperes and voltage levels of several thousand volts.
The total discharge energy is small, somewhere near the milli-Joule (mJ).
The power on the other hand is extremely high because the waveforms are changing extremely rapidly.

% What generates an ESD
Objects can accumulate electrostatic potential by tribocharging or electrostatic induction.
In the case of tribocharging, two objects are put into contact then separated.
The separation rips off charges from one another, causing one object to be positively charged and the other negatively.
%TODO: Depends on material properties - see table from ESD
Electrostatic induction works differently but the outcome is similar.
When a piece of material is placed inside an electric field, positive and negative charges of the material become spatially separated, at a macroscopic scale.
If the object is temporarily grounded in one point, charges near that location are evacuated, in majority positive or negative.
Similaryly to tribocharging, electrostatic induction causes the material looses its electrical neutrality.
For both phenomena, an electrostatic discharge happens when the potential difference between two near objects has become sufficiently large.
Both items are forced to return to electrical neutrality.

% How are they generated in the automotive field - Triboelectrification between tires and road surface
Tribocharging happens constantly when a vehicle is in motion \cite{generationESDautomotive}.
It occurs because of the friction of the tires on the road surface, which is a constant contact followed by separation mechanism.
Therefore, static electricity accumulates constantly inside a vehicle, until a discharge happens.

% Triboelectrification between a human body with clothing and the seat fabric
Triboelectrification may also happen between a human body with clothing and the seat fabric, eventually leading to an electrostatic discharge.
The probability of discharge is rather low \cite{generationESDautomotive} and the phenomenon can be only be observed in rare situations when the human body leaves the vehicle.

% Triboelectrification between body and flow of air
%TODO: Find reference for this phenomenon
Similarly, triboelectrification can also occur between the vehicle's body and the airflow generated by the motion.
More specifically, it is not the air itself but the particles carried in it that induce the contact and separation process.
