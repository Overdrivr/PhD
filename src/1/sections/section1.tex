\section{Electrostatic discharges in the automotive field}
\subsection{General concept}

% What is an ESD
\gls{esd} is the sudden flow of electricity between two differently-charged objects.
It is the result of the accumulation of electrostatic potential through tribocharging or electrostatic induction.
The discharge itself is caused either by contact, electrical short or dielectric breakdown.
Through static electricity buildup, thousands of volts can be accumulated locally, until a discharge happens.

% What are its key properties
\gls{esd} events have very short timespan, usually in the range of a few hundred nanoseconds.
Although very short, it is frequently observes flowing currents in the range of tens of amperes.
As a result, the overall discharge energy is small, but the power is extremely high.

% Impact
Such harsh events are very harmful for electronic systems and integrated circuits.
Today, there is not a single device in the field that is not protected against electrostatic discharges.
Due to the very short, fast, and random nature of \gls{esd} events, it remains very challenging to perfectly protect an electronic system against them.

\subsection{Generation mechanism in the automotive field}

% Harsh automotive field is harsh
From a general perspective, the automotive field is a harsh environment for electronic devices.
Equipements are exposed to a wide range of mechanical, electrical and thermal stresses.
A running engine generates plenty of vibrations, that can wear out electrical contacts, solder joints and connections.
It also generates heat, and a car can be exposed to large temperature variations during its lifetime.
Integrated circuits must be designed with those constraints in mind.

%TODO: References
% ... Especially for ESD
The automotive field has also the harshest requirements for \gls{esd}.
When a car is moving on the road, two different mechanisms generate static electricity accumulation.
Electronic devices inside a car are therefore constantly exposed to electrostatic discharges.

% Detail the first mechanism
When the car is moving on the road, the friction between the rubber wheels and the road generates tribo-electricity.
Basically, charges are stripped-off between the wheels and the road, leading to the accumulation of charges on the vehicle.

% Detail the second mechanism
Similarly, the friction between the car's body and the air flowing through it generates charge accumulation on the vehicle.

% How does the discharge happens
When electrostatic potentials become large enough, dielectric barrier breaks down and and a discharge happens.
It propagates through the car's body, wiring and electronic equipments.
