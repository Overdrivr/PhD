\section{State of the art in silicon-level operating ESD investigation}

% Transition ?
Section \ref{sec:state-art-esd-testing}
Introduction to generation of ESD events in laboratories.

% Recall problematic
How those events propagate inside electronic equipment, reach \gls{IC} and cause them to malfunction.

How to analyze and predict early that an \gls{IC} in operation has weaknesses against \gls{esd}.
This section, review of the litterature
Study cases
Observation methods
Modeling approaches to debug the behavior of integrated circuits in operation when they are exposed to \gls{esd}.

\subsection{Observation methods for failure detection and localisation}

EMMI
Near-field scan

\subsection{Study cases}

%TODO
Two soft-failure cases in \cite{systemLevelESDIC}.
%TODO: Describe cases
EMission Microscopy (EMMI) technique is employed to locate the part of the die causing for the soft-failure.
The chip is debugged and successful solutions are set in place to increase soft-robustness of the chip.

%TODO
One failure case in \cite{LacrampeTransientImmunity}
inject a disturbance on the power rail of a digital IC
accurately the voltage variations on the input and output.
Injection on the IC in operation is performed by a 1 nF capacitor similarly to DPI standard.

%TODO: What is the failure signature, provided fix, conclusions
Soft-failures on a SDRAM memory in operation \cite{SDRAMCase}.

%TODO:
Generation of soft-failures on a camera communication bus with different severities \cite{softFailSubsystem}.
Identify what element on the bus is causing an error, between the sensor or the application processor
Approach tested for localizing soft-failure is detection of significant variation in the emission spectrum of each component.
Emission spectrum and map is recorded with near-field magnetic scanner
Technique turned out inconclusive

Study of soft-failures on an LCD display \cite{softFailLCD}.
Stressing of the device is done with an IEC 61000-4-2 generator
Unsuccessful in identifying the root cause of error
Near-field scanning employed to identify which trace of the LCD's flex connector metal track is the most sensitive to disturbances
Unsuccessful because of the lack of spatial resolution of the probe
Repeated again with capacitively coupled TLP on each individual metal track but unconclusive results again.

%TODO: re-read article
Search of ESD propagation paths responsible for soft-failures on a mobile phone \cite{softFailMobile}.
Metallic paths are assumed to be main propagation paths
RC-networks used as countermeasures to protect physical inputs and outputs, like buttons, LCD inputs, connectors, etc.
IEC 61000-4-2 is used as stress source.

\subsection{Particular test methods}

%
Undersized TEM (TODO: GLOSS) cell for more compact and higher electrical fields \cite{SDRAMCase}
Filtered TLP similar to IEC 61000-4-2 \cite{iec61000-4-2} as stress generator for the TEM cell.

%
Near-field scanner as a very local stress injection system
Combined with a VF-TLP in \cite{NearFieldInjectionFabrice}.
Possible to map quantitatively and with good resolution ESD susceptibility of a board

%
Near-field scanner is also used in \cite{NearFieldInjectionBis} to identify metal tracks connected to ESD sensitive \gls{ic} inputs or outputs on a \gls{pcb}.


\subsection{Modeling methods of soft-failures for integrated circuits}

%
Mixed mode ESD simulations \cite{mixedModeESDSims}.

%
SEED methodology ? in \cite{usb2ESDProtection}.
Interactions between external devices leading to soft-failures and how to fix them ?

% TODO: What is the goal ?
3D EM simulation \cite{LacrampeTransientImmunity} at silicon level, with the integrated circuit layout.
Deduce capacitive couplings between Vdd and Vss rails
Model by equivalent impedance
Method is complex and long to simulate ? but was proven to work effectively to model parasitic couplings.
Alternative to parasitic device extraction directly from layout.
IBIS package data was employed to model the package by an RLC equivalent

%
Modelling of a TLP stress generator in \cite{LacrampeTransientImmunity} using a lookup table I(V) component, in series with a 50\textOmega{} resistor.

% TODO: Interest of the EM simulation ?
Electromagnetic fullwave simulations are conducted in \cite{softFailMobile}.
System-level components are simulated, such as PCB, metallic casing and battery back.

% Modelling of IC input pin from a TLP characterization
Proven in \cite{usb2ESDProtection} that a TLP characterization I(V) curve of an IC pin can be used to model it for a (signal ?) kind of input.

% IBIS is not enough for modelling an IC pin for ESD simulations
Input Output Buffer Information Specification (IBIS) \cite{ibis-spec} is a method for digital circuits providing the I/V characteristic of an IC without disclosing any circuit or process information.
IBIS provides a behavioral modeling technique for digital circuits for performing signal integrity simulation.
The idea is to characterize and distribute information about the package, buffers and protection diodes (not ESD protections), not the IC core.
It is demonstrated in \cite{ibisImprovementFabrice} that for EMC and ESD simulations the model is not sufficient.
It is not defined for fast impulses and high injection.
