\section{State of the art in silicon-level operating ESD investigation}

Section \ref{sec:state-art-esd-testing} introduced major \gls{esd} standards used in laboratories to test and qualify electronic products.
Sometimes, discharges propagate inside electronic equipments, reach the integrated circuits and cause them to malfunction.
It is essential to detect early functionnal weaknesses against \gls{esd}.
This section reviews the litterature on the current state of the art on operating \gls{esd} analysis.
Study cases, observation methods and modeling approaches are presented.
Ultimately, the goal is to debug efficiently the behavior of integrated circuits in operation when they are exposed to \gls{esd}.

\subsection{Study cases}

% Case 1 - NXP bandgap + substrate coupling
The case of an alternator's regulator \gls{ic} is presented in \cite{softfailEMCIC}.
where a product is investigated and designed for high immunity against electromagnetic disturbances.
The failure signature is a loss of the regulated voltage.
It was proven that the disturbance was coupling through the substrate on a current mirror inside the bandgap.
During the disturbance, bandgap voltage was shifting from its nominal value, causing a full system restart.
A design fix was proposed by filtering at the appropriate spot inside the design to avoid the amplification of the disturbance.
%TODO: Investigation method ?

% Case 2 - CESAME IC - paper Vrignon + Ben dhia
Another failure case is presented in \cite{LacrampeTransientImmunity}.
Very-fast \gls{tlp} is injected on an 0.18 \textmu{}m CMOS technology (1.8 V supply voltage) chip.
The chip contains 6 instances on the same logic core, differing by their power-rails architecture.
The injection on the power rail is performed by a 1 nF capacitor, similarly to the \gls{dpi} standard \cite{iec62132-4}.
% What is the failure signature
An output signal of the logic core is monitored.
The susceptibility criteria is the amplitude crossing a 20\% threshold from the established logic level.
Above this threshold, the core is supposed to no longer work reliably.
It is proven that modelling the output buffer of the core logic can reproduce with less than 20\% error the waveform on the output.
It is less accurate than a full-netlist simulation, but faster to simulate.
VHDL-AMS and SPICE modelling are performed in this analysis.

% Case 3 - failures on an SDRAM
%TODO: Read reference articles in this article
In \cite{SDRAMCase}, soft-failures are studied on a SDRAM memory in operation.
The injection setup consists in a modified \gls{tem} cell with a reduced septum height for increased field strenghts.
The SDRAM chip is mounted on a board.
Data is written and read on the memory by \gls{fpga}.
Differences between incoming and outgoing data signifies a functional failure of the memory.
Only the memory is exposed to the disturbance, the rest of the board's devices are located outside of the \gls{tem} cell, on the other side of the board.
The main defect of this method is to only provide a global failure level.
It does not allow to identify which particular net or pin is the most sensitive to disturbances.

In \cite{softFailSubsystem}, electrostatic discharges cause errors on two different camera communication buses.
Events of different severities are observed, depending on the discharge parameters.
It is attempted to determine whether the sensor or the application processor is causing the error.
The magnetic emission map is recorded with a near-field magnetic scanner to try to observe local variations in the emission spectrum because of the degradation of functionnality.
It was envisionned that soft-failure can induce significant variations in the emission spectrum of a disturbed component, and thus those variations could help localize them.
In this particular case, the root cause of failures could not be determined.

%TODO:
An LCD display is studied in \cite{softFailLCD}.
The device is tested with an IEC 61000-4-2 \cite{iec61000-4-2} generator, and errors are observed due to the discharge.
Non-destrutive problems
stripes on the display, optical parameter changes and blacklighting malfunction
System-level testing too complex for identifying root cause.
A near-field scan was performed to identify which trace of the LCD's flex connector has the highest sensitivity to disturbances.
This second approach was not successful either because of the lack of spatial resolution of the probe.
Finally, the individual track stressing was repeated with a capacitively-coupled \gls{tlp} on each individual metal track.
However, results were once more unconclusive and no metal trace could categorically be identified as more sensitive than the others.
Conclusion is call for IC level soft-error models.

%TODO: re-read article
Search of ESD propagation paths responsible for soft-failures on a mobile phone \cite{softFailMobile}.
Metallic paths are assumed to be main propagation paths
RC-networks used as countermeasures to protect physical inputs and outputs, like buttons, LCD inputs, connectors, etc.
IEC 61000-4-2 is used as stress source.

\subsection{Observation methods for failure detection and localisation}

\subsubsection{Emmission Microscopy (EMMI)}
%TODO: References
\gls{emmi}

%
EMMI is presented in \cite{softfailEMMI} as a successful analysis method for debugging and locating a functional failure on an integrated circuit.
\gls{emmi} is employed to locate the part of the die causing for the soft-failure.

\subsubsection{Near-field scanner}
%TODO: References
Near-field scan


\subsection{Particular test methods}

In the litterature, some particular test methods have been presented.
This section gathers a few of them that are relevant for powered-on \gls{esd} testing.

%
Undersized \gls{tem} cell for more compact and higher electrical fields \cite{SDRAMCase}
Filtered TLP similar to IEC 61000-4-2 \cite{iec61000-4-2} as stress generator for the \gls{tem} cell.

%
Near-field scanner as a very local stress injection system
Combined with a VF-TLP in \cite{NearFieldInjectionFabrice}.
Possible to map quantitatively and with good resolution ESD susceptibility of a board

%
Near-field scanner is also used in \cite{NearFieldInjectionBis} to identify metal tracks connected to ESD sensitive \gls{ic} inputs or outputs on a \gls{pcb}.


\subsection{Modeling methods of soft-failures for integrated circuits}

%
Mixed mode ESD simulations \cite{mixedModeESDSims}.

%
SEED methodology ? in \cite{usb2ESDProtection}.
Interactions between external devices leading to soft-failures and how to fix them ?

% TODO: What is the goal ?
3D EM simulation \cite{LacrampeTransientImmunity} at silicon level, with the integrated circuit layout.
Deduce capacitive couplings between Vdd and Vss rails
Model by equivalent impedance
Method is complex and long to simulate ? but was proven to work effectively to model parasitic couplings.
Alternative to parasitic device extraction directly from layout.
IBIS package data was employed to model the package by an RLC equivalent

%
Modelling of a TLP stress generator in \cite{LacrampeTransientImmunity} using a lookup table I(V) component, in series with a 50\textOmega{} resistor.

% TODO: Interest of the EM simulation ?
Electromagnetic fullwave simulations are conducted in \cite{softFailMobile}.
System-level components are simulated, such as PCB, metallic casing and battery back.

% Modelling of IC input pin from a TLP characterization
Proven in \cite{usb2ESDProtection} that a TLP characterization I(V) curve of an IC pin can be used to model it for a (signal ?) kind of input.

% IBIS is not enough for modelling an IC pin for ESD simulations
Input Output Buffer Information Specification (IBIS) \cite{ibis-spec} is a method for digital circuits providing the I/V characteristic of an IC without disclosing any circuit or process information.
IBIS provides a behavioral modeling technique for digital circuits for performing signal integrity simulation.
The idea is to characterize and distribute information about the package, buffers and protection diodes (not ESD protections), not the IC core.
It is demonstrated in \cite{ibisImprovementFabrice} that for EMC and ESD simulations the model is not sufficient.
It is not defined for fast impulses and high injection.
