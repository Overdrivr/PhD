\section{Research problem}
\subsection{Impact of ESD on electronic devices}

Hardware failure, at equipment level, at silicon level (gate oxide breakdown, short, etc)
Software failure

\subsection{Identification and modeling of design weaknesses against ESD}

Core question

\subsection{Research approach}

The category of systems we are dealing with here are purely hierarchical.
The final failure signature may be located at the silicon-level of an integrated circuit.
This IC is part of an electronic equipment, itself part of a multitude of devices connected together,
that are in charge of several functionnalities of a vehicle.

SCHEMA

ESD (vehicle level) -> communication bus, power line -> electronic equipment ->
PCB -> Integrated circuit -> FAILURE -> propagates back, with consequences up to vehicle level

*SIMULATION*

It is important to estimate properly the amount of stress a particular integrated circuits will be exposed to.
Appropriate models need to be derived for running electrical simulations at the system level (everything up to the integrated circuit pins).
Those models need to comply with the timescales, voltage and current dynamics, frequency behaviors, etc, encountered during electrostatic discharges.

This system level modeling methodology is the first step in this research work.

Once we are able to estimate which amount of stress is actually injected inside the IC, the objective is to estimate how the current propagates at the silicon-level.
Typically, at silicium level, a part of the ESD can be absorbed by ESD protections, or injected inside the substrate through parasitic coupling, etc.

In this second part, it will be necessary to estimate the accuracy of the existing device models at silicium level, provided by technology foundries.
Missing elements, such as layout resistance, parasitic couplings, etc, will need to be identified in order to try to achieve accurate ESD simulations at ESD level.
To validate the methodology, measurement data will need to be gathered at silicium level, which will require particular methods to be obtained.

Once done, both simulation methodology at system and silicium level will constitute a first tool for soft-failure investigation.

*INVESTIGATION*

The investigation focuses on the behavior of the integrated circuit and tries to answer the following.
How a 100 ns electrical event can disturb an integrated circuit to the point where the entire system is no longer fullfilling its job for a notable period of time ?

For this investigation, there is a clear relationship between the duration of the ESD and the duration where the function is no longer in specification.
In itself, an ESD is an extremely short even, almost always to the point where it negligible compare to the time constants of the studied system.
However, (litterature ?), this short event can cause functions to shut down in cascade, leading to an effective multiplication of the disturbance duration.
How to predict this behavior, and what options can be used to avoid ?

Real-cases will be detailed to understand this problem.

*MODELING*

The multi-level characteristic of the studied system will also call for simplified IC models.
Given today's circuit simulator performance in terms of speed and convergence, it is unrealistic to hope to achieve full system simulation.
Simplified IC models will either target higher-order modeling of functions performed by the IC, or modeling in terms of IO (input/output impedance).
A failure model that will raise an error during simulation given some conditions are met will also probably be required.
