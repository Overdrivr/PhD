%\section{Research problem}
\subsection{Impact of ESD on electronic devices}

%TODO: Pictures + References

% Hardware failure - how does it happen
%TODO: Detail more from article
Because of high voltage, current and power levels involved, electrostatic discharges can destroy electronic devices.

Inputs and outputs require \gls{esd} protections to route the discharge current outside the chip before it reaches the core circuit.
When the protection fails to work, the device usually gets destroyed and different failure signatures can be observed.
Failures may also occur at the board or equipment level of an electronic system.

% Gate oxyde breakdown
Very large voltage variations (high $dV/dt$) can cause a gate oxyde breakdown.
In this scenario, the local gate voltage temporarily exceeds the oxyde breakdown voltage.
The transistor gate that normally acts as an insulator suddenly leaks more current.
The device is considered destroyed because normal operation can no longer be fullfilled.

% Thermal breakdown
Long \gls{esd} pulses can induce significant and very localized device heating.
If the local temperature exceeds the silicon (or the material's) melting point, the device will locally melt.
Leakage current will increase, or parts of the device become short-circuited.
This failure signature is called a thermal breakdown.

% Others
%TODO: metal track fusing ?
Hardware failures can also occur at the equipment or board level.
Usually, they will involve the destruction of some discrete devices that make the board no longer working.


% Types of soft failures
Diverse consequences can be observed when a device is disturbed in its functionnality by an \gls{esd}.
They can be categorized in diverse ways.
Some usual functionnal failures are described hereafter.
In the less severe scenario, the functionnality is disturbed just for the duration of the ESD and recovers immediately after without noticeable consequence.
The next level is a circuit that performs a restart because the \gls{esd} disturbed some critial nets or parameters.
This is common when supply voltages are disturbed for instance, and the supply startup function understands the disturbance as a normal device power-up.
The device can also perform a restart because proper operations cannot be ensured due to the disturbance, and a recovery is attempted.
In most cases, the device recovers normal operation after a delay, but this behavior is not necessarily wanted.
It is an issue when an unwanted and unpredictable restart happens at the moment where the \gls{ic} must perform highly critical functions such as triggering airbags of a car.
With the third level of severity, the system gets frozen or stuck into an unwanted state because of the \gls{esd}.
The only way to recover normal operation requires a user-intervention.
User intervention can take the form of turning the key to shut down and restart the vehicle's engine.
Finally, \gls{esd} can put a system into a non-recoverable state because of a hardware failure.
This last level is not considered for functional robustness analysis.

% What is the challenge
In the given context, functional failures represent a risk just as important as hardware failures.
To limit risks and costs inherents to upgrading a device after it was deployed in the field, these failures must be taken care of as early as possible.
Ideally, the robustness of an electronic product should be studied, characterized or simulated during its design phase.
The research conducted and presented in this document aims to develop new tools and techniques for studying and predicting functional failures.

\subsection{ESD protection}

What is an ESD protection
%TODO: Example snapback/no snapback I(V) curve
I(V) curves
Integrated or discrete
Modelling is presented in chapter X

\subsection{Vrac}


% Presentation of the structure of the research/document
The litterature is studied in section \ref{sec:state-art-esd-testing} to identify the current state of the art in powered \gls{esd}.
This preliminary work highlights how functional failures appear and how they impact electronic devices.
It is demonstrated that so far integrated circuits are studied as black-box objects for studying functionnal failures.
Stresses are injected on external inputs while external outputs are monitored for failures.
The amount of silicon-level studies and research remains small.

% What is presented in this document ? Investigation and detection methods
In this research work, the goal is to study functional failures inside the integrated circuit, at silicon level.
Several methods are proposed for detecting potential weaknesses early.
Each of them is evaluated against a study-case that has a few well-known weaknesses.

% Fix and correction of design is outside scope of this document
The topic of fixing potential weaknesses is outside the scope of this document.
At silicon-level, designers can fix the issues given that appropriate information are provided.
This study is focused on understanding how failures appear, propagate and can be detected, and not on how to fix them.

% First part of the study : understand what reaches the IC
The first challenge for understanding soft-failures at silicon-level is to determine what fraction of an incoming \gls{esd} actually reaches the integrated circuit.
Between the injection point of a disturbance in a system and the disturbed integrated circuit, many devices will be connected such as connectors, passive and active devices, metal traces.
Each element interacts with the discharge, can absorb a part of its current or change the waveform.

% Use simulations as investigation tool
Electrical simulators constitute the main investigation tool of this study.
For \gls{esd} simulations to be accurate, appropriate models need to be derived.
Building models that work both in static conditions and for very fast, very short and very strong electrical stresses is not straightforward.
Those models need to comply with the timescales, amplitudes and frequencies encountered during electrostatic discharges.
Chapter \ref{sec:esd-modeling} details a modeling method that lays the basis to achieve accurate simulations.

% Second step is to study and measure what happens inside the IC
An \gls{esd} induced failure case of an integrated is studied in the next chapter \ref{sec:study-real-product}.
In a first phase, measurement data is obtained at the board level and the failure is explained.
Simulations are run to understand how the failure appears, and more specifically how a short electrical event can disturb an integrated circuit for a long period of time.
In a second phase, the integrated function is placed onto a custom testchip.
It embarks specific structures to obtain measurement data on electrical nets that are not physically accessible.
All these measurements are performed for the purpose of estimating the accuracy of integrated circuit \gls{esd} simulations.
There are two main potential sources of error that are checked.
Silicon technology device models are not designed to function for extremely fast transient transient disturbances.
Also, standard simulations do not take parasitic devices into account, such as metal track resistances and parasitic couplings.
Measurement data is confronted to simulations in order to verify the validity of models.

%
Analysis of the issue is performed manually, by trial and error, searching inside the design why the function is failing.
It is a complex and time-consuming process.
Also, this approach is almost exclusively possible when searching for a particular weakness.
The core research of this work focuses on new analysis methods proposed in chapter \ref{sec:methods-operating-esd-analysis}.

%
Some part of the research work also focused towards \gls{esd} testers.
System-level \gls{esd} guns \cite{iec61000-4-2, iso10605} constitute a major testing and qualification tool, but bring too much complexity during silicon-level investigation.
To overcome this issue, a new stress generator based on a \gls{tlp} is proposed in chapter \ref{sec:tlp-hmm}.
It produces the same compliant waveform than those standards, but with some interesting advantages.
It is not a drop-in replacement but can be interesting as an investigation tool.

%
Final conclusions and potential following work are detailed in chapter \ref{sec:final-conclusion}.
