\subsection{Impact of ESD on electronic devices}

%TODO: Pictures + References
% Are electronic devices exposed to ESD
As detailed in the introduction, electrostatic discharges constitute a large threat for electronic devices.
Failures can occur during manufacturing.
The manipulation of parts by manufacturing machines involves repeated contact and causes triboelectrification.
Ultimately, discharges happen and devices get destroyed.
Over the years, manufacturing processes and standards have improved, reducing the requirements of discharge levels to sustain.
In parallel, factory environment were better studied to identify the actual levels devices are exposed to.
The goal is to reduce the pressure on semiconductor designers that need larger silicon area to protect devices, because of the shrink of technologies and robustness.
%TODO: Manufacturing and ic-level Standards
Failures also happen when the device is released in the field and exposed to its operating environment.
Manipulation by electrically-charged humans a is major source of danger for commercial products like cellphones and cameras.
The automotive environment is even harsher as detailed in the global introduction, with vehicles being a major source of electrostatic discharges.

% Main impact is hard-failure
A hard-failure corresponds to changes in the material structure or properties of a device to the point where it no longer works.
\gls{esd} induce those changes because of the extremely large current densities, high voltages, and power levels involved.
More specifically, different kinds of failure signatures can be observed.
Pictures of failures obtained with an electronic microscope are provided in Fig. \ref{fig:silicon-level-failures}.
%TODO: Comment

\begin{figure}[!h]
  \centering
  \includegraphics[width=0.3\textwidth]{src/1/figures/example_silicon_failures.pdf}
  \caption{Example of ESD induced failures at silicon-level}
  \label{fig:silicon-level-failures}
\end{figure}


% Oxyde breakdown
%TODO: Etoffer avec these monnereau
%TODO: Values
Oxyde breakdown happens when the electric field is larger that an insulating material can tolerate.
With \gls{esd}, the big voltage variations in a short time can result in electric fields superior to 1kV/m ?
For reference, thunder and lightning events have electrical fields in the same order of magnitude.
Oxyde breakdown is frequently found in the insulator constituting the gate of a \gls{mos} transistor.
After the failure, the gate that is normally insulated from the rest of the devices leaks significant current.

%% Thermal breakdown
Thermal breakdown is the result of an elevation of temperature inside the silicon, above its melting point.
It is caused for long discharges that can induce significant and very localized device heating.
It results in a significant increase of leakage current or apparition of short-circuits.

Due to the very short, fast, and random nature of \gls{esd} events, it remains challenging to perfectly protect an electronic system against hard-failures.

% New impact is soft-failure
Diverse consequences can be observed when a device is disturbed in its functionnality by an \gls{esd}.
They can be categorized in diverse ways.
Some usual functionnal failures are described hereafter.
In the less severe scenario, the functionnality is disturbed just for the duration of the ESD and recovers immediately after without noticeable consequence.
The next level is a circuit that performs a restart because the \gls{esd} disturbed some critial nets or parameters.
This is common when supply voltages are disturbed for instance, and the supply startup function understands the disturbance as a normal device power-up.
The device can also perform a restart because proper operations cannot be ensured due to the disturbance, and a recovery is attempted.
In most cases, the device recovers normal operation after a delay, but this behavior is not necessarily wanted.
It is an issue when an unwanted and unpredictable restart happens at the moment where the \gls{ic} must perform highly critical functions such as triggering airbags of a car.
With the third level of severity, the system gets frozen or stuck into an unwanted state because of the \gls{esd}.
The only way to recover normal operation requires a user-intervention.
User intervention can take the form of turning the key to shut down and restart the vehicle's engine.
Finally, \gls{esd} can put a system into a non-recoverable state because of a hardware failure.
This last level is not considered for functional robustness analysis.

% What is the challenge
In the given context, functional failures represent a risk just as important as hardware failures.
To limit risks and costs inherents to upgrading a device after it was deployed in the field, these failures must be taken care of as early as possible.
Ideally, the robustness of an electronic product should be studied, characterized or simulated during its design phase.
The research conducted and presented in this document aims to develop new tools and techniques for studying and predicting functional failures.

All devices are protected against electrostatic discharges using specific measures, otherwise they would not withstand it.

\subsection{ESD protection}

ESD protections are one of the main preventive measures to protect circuits

% Principe de fonctionnement
Deviate current before it reaches sensitive circuitry
ESD are made to work in transient regime
No ESD protection can sustain the current levels they are defined for for an unlimited amount of time.

% Types of protection
On-chip or discrete, external

% Requirements
Speed ?
Cost is relevant, but tradeoff between on-chip and off-chip
External protections increase the BOM and cost.
But on-chip protections too.
Cost of silicon area is extremely elevated.
High stakes to develop ESD protections able to sustain specifications and requirements in the tighest possible space.
ESD protections must be transparent to the rest of the device.
True when exposed to ESD, but also when not.
Two boudaries
%TODO SChema
The lower boudary corresponds to the operating levels of the device.
Example is a 5V IO.
If ESD triggers at 4V, it will trigger during normal operation and be immediately destroyed because it cannot sustain levels in DC or low frequency.
On the other hand, the ESD protection must trigger below the Safe Operating Area limits.
If the integrated technology can only withstand voltages below 50V even for a very short amount of time, and the ESD triggers at 60V, the circuit will be destroyed.
ESD protections thus are designed to operate between those two boundaries.

% Shift of sustaining system-level stress
Over the years, the responsibility of sustaining system-level stresses has shifted from external devices and PCB to the integrated circuit on-chip protections.
Motivated by economical reasons.
Very challenging for on-chip ESD designers
The operating area of an ESD drastically reduces, requiring protections to have smaller on-resistance, to trigger at given voltage accurately, independently of manufacturing process, layers mismatches and temperature variation.
High competitivity domain
Rely on tools like Centaurus TCAD to simulate semiconductor physics and predict what how an ESD protection will behave.

% Common architectures

Diode


Thyristor

%TODO
\begin{figure}[!h]
  \centering
  \includegraphics[width=0.3\textwidth]{src/1/figures/iv_curves_esd_protections.pdf}
  \caption{I(V) curves of typical ESD protections with snapback and no-snapback}
  \label{fig:iv-curve-esd-protection}
\end{figure}

Thyristor often have snapback waveforms
Particular part of the structure triggers on when exposed to sufficiently large amplitudes
Explains why the curve has two slopes

RC-triggered MOS
Power transistor activated by an RC network.
C is most of the time the parasitic gate capacitance to supply ?
In high-regime, the C acts as a short, rising the gate potential.
The MOS switches on and absorbs current, deviating it into the ground.

I(V) curves



Modelling is presented in chapter X
