% THIS IS NOT USED INSIDE THE DOCUMENT - NOT INTERESTING ENOUGH
\subsection{Assembling models}

% Making a library of models is not the panacea - experience for assembling them is also required
So far, only described individual elements that constitute a model library.
When modeling an entire system, it is unrealistic to expect every single metal track, piece of wire, or parasitic device to be modeled.
it would take too much time and is impossible to identify them all.
Also, some of those elements can be ignored because not all of them impact simulations with the same weight.

%
The modeling method consists in connecting together models of devices that physically exist.
To reach good agreement between simulation and measurement, some phenomena must also be taken into account.
They can strongly impact waveforms.
In particular, propagation and reflection phenomena play a key role at the nanosecond timescale with macro-scale systems.
They are completely ignored in standard time-domain electrical simulations and analysis, because the time constants involved are much shorter than the typical timescale.
For instance, a combination of a propagation delay with reflected wave sometimes leads to dampened oscillations with a period in the range of a few nanoseconds that last for tens of nanoseconds.
In comparison, most analog functions in the automotive field have time constants above tens of microseconds, that is to say 5 orders of magnitude longer.
In those simulations, nanosecond scale event can be purely and simply ignored.
However, in \gls{esd} simulations, nanosecond scale event are important because the waveforms last for the same order of magnitude.

% Propagation phemenon
Propagation is a phenomenon that always occurs but is negligible over a few tens microseconds.
Below that timescale it must be taken into account because it has a major impact on the waveforms.
This is due to the fact that observation times are in the same range than electrical wave propagation times.

% Reflection phenomenon
Reflection happens when an electrical wave propagates and reaches a point where two propagation media are mismatched (two coaxial cables of different characteristic impedance for example).

% Combined impact
These two effects and in combination can impact waveform in some specific scenario

% Measurement artifact
Spike visible on measurements but is just an artifact
Caused by a delay between an intended measurement point and where the measurement is actually performed
Delay because of a cable for instance
Setup given in Fig \ref{fig:setup-measurement-spike}

\begin{figure}[!h]
  \centering
  \includegraphics[width=0.3\textwidth]{src/1/figures/setup_measurement_spike.pdf}
  \caption{Typical setup causing a measurement artifact}
  \label{fig:setup-measurement-spike}
\end{figure}

The process that generates this measurement glitch starts with the injection of a fast rectangular pulse.
The pulse propagates towards point A (step 1).

\begin{figure}[!h]
  \centering
  \includegraphics[width=0.3\textwidth]{src/1/figures/spike_generation_1.pdf}
  \caption{Spike generation - step 1}
  \label{fig:spike-step-1}
\end{figure}

At the moment it reaches A, the capacitor load is not immediately visible.
It is “hidden” from point A by the short delay.
Thus, the voltage rises, depending at this point only on the characteristic impedances of the cables.
A part of the pulse propagates towards the load and a part toward Vout (depending on the impedance ratio) (step 2).

\begin{figure}[!h]
  \centering
  \includegraphics[width=0.3\textwidth]{src/1/figures/spike_generation_2.pdf}
  \caption{Spike generation - step 2}
  \label{fig:spike-step-2}
\end{figure}

The impulse reaches then the load (step 3), and propagates back toward A, settling the voltage at A to
V – (V – Vload) = Vload (forward traveling wave minus reflected wave from load) (step 3).
The capacitor is supposed to be charging at constant current, voltage rises following a linear curve.

\begin{figure}[!h]
  \centering
  \includegraphics[width=0.3\textwidth]{src/1/figures/spike_generation_3.pdf}
  \caption{Spike generation - step 3}
  \label{fig:spike-step-3}
\end{figure}

Now the voltage at point A is defined by the load.
However before that, the voltage rose and generated a peak that will hit Vout after a delay.
This peak preceding the capacitor charge is detailed on figure \ref{fig:spike-step-4}.
The amplitude is exaggerated for illustration purposes

\begin{figure}[!h]
  \centering
  \includegraphics[width=0.3\textwidth]{src/1/figures/spike_generation_4.pdf}
  \caption{Spike generation - step 4}
  \label{fig:spike-step-4}
\end{figure}

In a more realistic situation, if the short delay is of the same order of magnitude than the pulse risetime, the generated peak will be a non-negligible fraction of the initial pulse amplitude.

\begin{figure}[!h]
  \centering
  \includegraphics[width=0.3\textwidth]{src/1/figures/spike_generation_5.pdf}
  \caption{Measured waveform}
  \label{fig:spike-step-4}
\end{figure}

% Unterminated cables
Because of propagation and reflection phenomena, elements that are normally ignored in standard simulations can no longer be neglected.
A perfect illustration of this point are cables connected to the circuit on one end, and left floating on the other end (or in high impedance)
In regular simulations, they can removed entirely.
With ESD simulations, they will induce oscillations and amplitude changes

\begin{figure}[!h]
  \centering
  \includegraphics[width=0.3\textwidth]{src/1/figures/setup_unconnected_cable.pdf}
  \caption{Typical setup causing oscillations}
  \label{fig:setup-unconnected-cable}
\end{figure}

Transient current propagates toward the unconnected end
Reflects entirely at the high impedance termination
Comes back inside the circuit, absorbing a part of the initial current and supplying it later on.
The consequence is an undershoot followed by a delayed overshoot (fig. 15).

\gls{tlp} is a perfect example of an application exploiting an unterminated cable

% Feeding cables
Similarly, all cables on the propagation path (Fig. \ref{fig:setup-feeding-cable}) must not be neglected just because they are matched with the circuit.
The delay they introduce impacts greatly the waveforms.
TLP generators always require a cable to connect the load under test to the generator.
This connection cable plays a role in the final waveform.
In simulation, if omitted, the waveform is very rectangular and clean (Fig. \ref{fig:comparison-feeding-cable}).
When the cable is added, the presence of reflections with this delay changes quite a lot the waveform that now has an initial step between x and \SI{0}{\nano\second}, with an amplitude corresponding to the TLP charging voltage and not the circuit operating point
\SI{0}{\nano\second} corresponds exactly to the delay of the feeding coaxial cable.

\begin{figure}[!h]
  \centering
  \includegraphics[width=0.3\textwidth]{src/1/figures/setup_feeding_cable.pdf}
  \caption{Minimal TLP generator with feeding cable and a mismatched load}
  \label{fig:setup-feeding-cable}
\end{figure}

\begin{figure}[!h]
  \centering
  \includegraphics[width=0.3\textwidth]{src/1/figures/comparison_feeding_cable.png}
  \caption{Simulated waveform with and without feeding cable}
  \label{fig:comparison-feeding-cable}
\end{figure}

% Conclusion
Beyond these particular examples, the idea is to be cautious about even short delays when assembling models from the library.
