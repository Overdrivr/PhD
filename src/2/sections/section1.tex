\section{System-level modeling}

Simulations are a great tool to get a better understanding of the impact of \gls{esd}s on integrated circuits.
They allow introspection inside specific nets of the circuit, observe behavior that would be hard to observe in the field.
However, \gls{esd} simulations are not common practice and must be validated against measurement data before being used as an investigation tool.
Models for key devices must be derived, at each scale of the system.

In this chapter, the methodology for building models at the system level is detailed.
It mostly targets ESD generators and lab equipment.

\subsection{Lossless transmission lines}

\gls{esd} events generally last a few hundred nanoseconds.
At this timescale, delays introduced by usual coaxial cables found in labs and metal traces on \gls{pcb} are no longer negligible.
Depending on the configuration, these delays can generate oscillations, amplitude variations.

Most cables and PCB traces, although of very different construction, usually behave as lossless transmission lines.

A transmission line is defined by two key parameters, its \gls{Zc} and propagation delay.
Alone, these two parameters can be used in a behavioral model to describe efficiently and with great accuracy the behavior of lossless transmission lines.

The electrical model is constituted of two voltage-controlled voltage sources and two resistors.
Compared to the classic RLC distributed model, the behavioral model has infinite bandwith, and constant complexity (compared to RLC where the amount of elements increases with the required bandwidth).

ELECTRICAL MODEL

EQUATIONS
