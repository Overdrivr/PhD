\section{Conclusion}

% First section
In conclusion of this chapter, several tools have been presented for soft-failure analysis.
The system-level modeling method is very helpful to create accurate models, in order to simulate waveforms propagating up to an integrated circuit.
It was a first required step before being able to study the apparition of functional failures inside integrated circuits, at the silicon level.

% Second section
The propagation and reflection mechanisms studied were used later on to create a custom pulse generator.
This new generator, called TLP-HMM, reproduces the waveform of the HMM specification and IEC 61000-4-2/ISO 10605 standards.
Combined together, they define the most widely employed pulse in the entire ESD field.
The TLP-HMM brings several advantages compared to a standard HMM generator.
Among other things, it offers advanced discharge reproducibility, a fully shielded propagation path and zero radiated emission.
The current design is a prototype that helped identify improvements to make in future iterations.
In particular, higher charging capability is required.

% Failure correlation
To test this generator against actual ESD guns, a set of 10 different ESD protections was stressed and destroyed with a TLP, a TLP-HMM and an HMM generator.
The comparison of failure levels between all of them led to discover a correlation law using the simplest possible equivalent circuit.
It relies on calculating the failure current, using the equivalent generator impedance, the ESD protection on-resistance and the charging voltage.
Then, calculating this failure current for one generator allows to guess the failing charging voltage of any other.
Charging voltages could be accurately guessed on the entire set of 10 protections.

% Near-field scan
Finally, two post-processing methods for a near-field on-chip probe were detailed.
The first method relies on an integration of the measured signal in the time domain.
The second method processes the signal in the frequency domain, by compensating it with the sensor characterization.
Accuracy of each method was roughly estimated.
Future work will involve a better assessment of the accuracy, and improvements to the processing script.
