%
% Remi Beges PhD final report
%

% Glossaries
\newglossaryentry{Zc}
{
  name=characteristic impedance,
  description={is the ratio of the amplitudes of voltage and current of a
               single wave propagating along the line;
               that is, a wave travelling in one direction in the absence of
               reflections in the other direction.}
}

\newglossaryentry{computer}
{
  name=computer,
  description={is a programmable machine that receives input,
               stores and manipulates data, and provides
               output in a useful format}
}

\newglossaryentry{block}
{
  name=block,
  description={in the context of this document,
               a block is a fundamental part of an integrated circuit's function,
               usually composed of multiple devices and transistors,
               but that still provide a minimal, unit functionnality.}
}

\newglossaryentry{testbench}
{
  name=testbench,
  description={A test bench or testing workbench is an (often virtual)
               environment used to verify the correctness or soundness of a
               design or model, for example, that of a software product.
               (Source: Wikipedia)}
}

\newglossaryentry{dc}
{
  name=DC,
  description={Direct Current : Conditions where the circuit voltages
               and currents are independent of time}
}

\newglossaryentry{esd-gun}
{
  name=ESD gun,
  description={A device able to generate electro-static discharges in a
               laboratory environment for testing purposes.}
}

\newglossaryentry{esd-protection}
{
  name=ESD protection,
  description={Electronic device that protects sensitive electronics from
               electro-static discharges, usually by deviating the discharge
               current into a near ground.}
}

\newglossaryentry{soft-start}
{
  name=soft-start,
  description={Slow rise of the supply voltages, happening during device
  power-up, to protect sensitive integrated devices.}
}

\newglossaryentry{ldo}
{
  name=LDO,
  description={A low-dropout or LDO regulator is a DC linear voltage regulator
  that can regulate the output voltage even when the supply voltage
  is very close to the output voltage.}
}

\newacronym{bom}{BOM}{bill of materials}

% Acronyms
\newacronym{tlp}{TLP}{Transmission Line Pulsing}
\newacronym{esd}{ESD}{Electro-Static Discharge}
\newacronym{ic}{IC}{Integrated Circuit}
\newacronym{dut}{DUT}{Device Under Test}
\newacronym{pcb}{PCB}{Printed Circuit Board}
\newacronym{asic}{ASIC}{Application Specific Integrated Circuit}
\newacronym{io}{I/O}{Input and/or Output}
