\subsection{On-chip near-field current measurement and postprocessing}
\label{sec:on-chip-near-field-process}

%TODO: Mathematical concept ?

\subsubsection{Time-domain integration method}

% Describe the algorithm
The time-domain method relies on the fact that the measured voltage is proportionnal to the derivative of the coupled current.
The algorithm begins with integrating the measured voltage, using a basic trapezoidal integration.
With the calibration sensor, it is possible to determinate experimentally the gain of the sensor.
%TODO: How is it done ?
Indeed, 1A of current does not generate 1V on the sensor, but a much lower value.
The correction factor with the current design was found to be $8.10^8$.
As a result of the integration, an offset is present a $t=0$.
So far, it is simply given the value of the calibration measurement at $t=0$.

%TODO: Give the result

% Limitations
The time-domain method is simple to compute.
However, it makes several assumptions on the sensor.
Mainly, it assumes that the gain of the sensor is constant for all frequencies.

\subsubsection{Frequency-domain reconstruction method}

The frequency domain tries to improve the post-processing by taking the sensor's response into account.
First, the sensor must be characterized, using a virtual network analyzer (Put glossary) (sure ?).
%TODO: Talk about complex measurement required
Then, the inverse of the sensor's characterization in the frequency domain is calculated.
%TODO: Is it really the inverse ?
%TODO: Give the formula

%TODO: Setup and Curve

In parallel, the FFT (glossary) of the measured waveform to post-process is computed.
For the algorithm to work, a complex FFT must be employed.
It computes for each frequency the amplitude and phase of the signal.

Afterwards, the FFT of the signal to process is multiplied by the inverse of the sensor's characterization.
Since both are complex numbers, real and imaginary parts must be multiplied separately, following complex-numbers algebra rules (source).

%TODO: Formula

Finally, the inverse FFT of the result is calculated to bring back the waveform into the time-domain.

% Talk about discrete
All waveforms, either measured experimentally or simulated, are constituted by a set of discrete points.
As such, discrete FFT and IFFT must be employed.

% Talk about repo
This post-processing method was implemented in python (ref).
It is freely distributed as open-source software (ref).

\subsubsection{Comparison of both methods}
This waveform can be compared with the original TLP stress used during calibration, and the curve reconstituted with the integration method.
