\subsection{On-chip near-field current measurement and postprocessing}
\label{sec:on-chip-near-field-process}

% Introduction
In this section, two different methods are described for reconstructing the original current from a near-field magnetic scanner measurement.
Each method requires a characterization of the sensor, to obtain its transfer function.
Overall, they differ mostly in the modelisation of the transfer function.
The original time-domain waveform is obtained by using the measured waveform and the inverse transfer function.
%TODO: Say pretty standard
%TODO: Refs monnereau, review of nfs methods

\subsubsection{Physical concepts}
\label{sec:phys-concepts-nfs}

The sensor is a near-field magnetic current loop.
We are interested in the relation between voltage difference across the sensor and the current value circulating through the track.

% Creation of the magnetic field
Circulating current in a conductor creates a magnetic field.
In steady-state conditions (static current), and by approximating the shape of the metal track to an infinitely long wire, Biot-savart law could be employed (Eq. \ref{eq:biot-savart}).
It states that magnetic field decreases with the square of the distance from a current segment of the track.
\begin{equation}
  \label{eq:biot-savart}
  \overrightarrow{B}(\overrightarrow{r}) = \frac{\mu_{0}}{4\pi}\oint_{C}\frac{I \mathrm{d}\overrightarrow{l} \wedge (\overrightarrow{r} - \overrightarrow{r'}) }{|\overrightarrow{r} - \overrightarrow{r'}|^3}
\end{equation}

However Biot-Savart works only for magnetostatic approximation ? What is it ?
Steady state magnetic field.
General purpose is Jefimenko's equations

% Generation of voltage difference from the magnetic
The sensor is placed near the monitored metal track.
The sensor is approximated as a perfect loop in a magnetic field.
In that case, the Faraday law's of induction gives Eq. \ref{eq:faraday}
where \textepsilon{} is the electromotrice force (TODO: gloss) and \textPhi{}\textsubscript{B} is given in \ref{eq:phi}.

\begin{equation} \label{eq:faraday}
  \varepsilon = - \frac{\mathrm{d} \phi _{B}}{\mathrm{d} t}
\end{equation}


\textPhi{}\textsubscript{B} is the magnetic flux through the loop, $B$ is the magnetic field, \textSigma{}(t) is a surface bounded by the closed contour,
and $dA$ is an infinitesimal vector area element of \textSigma{}(t) (magnitude is the area of the element, direction is orthogonal to the loop surface).

\begin{equation} \label{eq:phi}
  \phi _{B} = - \iint_{\Sigma(t)} \mathrm{d}A . B(r,t)
\end{equation}

% Global equation and conclusion
%TODO: eq
Original current and measured voltage are related by a time-domain derivative ? Only in magnetostatic conditions ?

\subsubsection{Time-domain integration method}

% Describe the algorithm
As detailed previously in \ref{sec:phys-concepts-nfs}, the measured voltage is proportional to the time derivative of the original current.
The time-domain method is based on this property.
The original current waveform can be found equal to the integral of the voltage by a scale factor.
The scale factor can be determined experimentally during calibration.

% Talk about calibration

% Integration method
The algorithm begins by integrating the measured voltage, using a basic trapezoidal integration.
%TODO: Why trapezoidal

% Talk about current sensor
The correction factor with the current design was found to be $8.10^8$.
As a result of the integration, an offset is present a $t=0$.
So far, it is simply given the value of the calibration measurement at $t=0$.
The reconstructed curve is compared to the original in Fig. \ref{fig:time-domain-reconstructed}.

\begin{figure}[!htbp]
  \centering
  \includegraphics{src/3/figures/time_method.pdf}
  \caption{Reference current waveform versus time-domain reconstructed}
  \label{fig:time-domain-reconstructed}
\end{figure}

% Limitations
The time-domain method is simple to compute.
However, it makes several approximations regarding the sensor's shape and the validity of the underlying physical laws for fast transient waveforms.
Overall, it assumes that the gain of the sensor is constant for all frequencies, which is not correct either.

\subsubsection{Frequency-domain reconstruction method}

%TODO: Say that it is general purpose
%TODO: Which approximations compared to physical principles
The frequency domain tries to improve the post-processing by taking the sensor's response into account.
First, the sensor must be characterized, using a virtual network analyzer (Put glossary) (sure ?).
%TODO: Talk about complex measurement required
Then, the inverse of the sensor's characterization in the frequency domain is calculated.
%TODO: Is it really the inverse ?
%TODO: Give the formula

%TODO: Setup and Curve

\begin{figure}[!htbp]
  \centering
  \includegraphics{src/3/figures/freq_method.pdf}
  \caption{Reference current waveform versus frequency-domain and time-domain reconstructions}
  \label{fig:freq-domain-reconstructed}
\end{figure}

In parallel, the FFT (glossary) of the measured waveform to post-process is computed.
For the algorithm to work, a complex FFT must be employed.
It computes for each frequency the amplitude and phase of the signal.

Afterwards, the FFT of the signal to process is multiplied by the inverse of the sensor's characterization.
Since both are complex numbers, real and imaginary parts must be multiplied separately, following complex-numbers algebra rules (source).

%TODO: Formula

Finally, the inverse FFT of the result is calculated to bring back the waveform into the time-domain.

% Talk about discrete
All waveforms, either measured experimentally or simulated, are constituted by a set of discrete points.
As such, discrete FFT and IFFT must be employed.


\subsubsection{Conclusion}
% Conclusion

% Limitations

% Code repository
This post-processing method was implemented in python language (ref).
It is freely distributed as open-source software (ref) under the MIT licensing (ref).
