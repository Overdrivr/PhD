\subsection{Test vehicle verification and testing after manufacturing}
\label{sec:test-vehicle-testing}

The testplan for the testchip is constituted of 7 steps.

\begin{enumerate}
\item Board supplies
\item Testchip supplies
\item Master-slave communication
\item Near field current measurement
\item Testchip communication
\item Monitoring system validation
\item ESD tests for regulation failure
\end{enumerate}

For steps 1 and 2, the supplies were simply checked with a voltmeter.
All regulators from both boards and inside the testchip were found to operate as expected.

\subsubsection{master-slave communication}

% What is this step
In step 3, the communication between the master board and the slave board is verified.
The goal is to ensure that signals transit correctly between the microcontroller \gls{io}s and the testchip \gls{io}s.
The goal is not however to validate that the data is correctly exchanged.
Only the physical layer is checked.

% What is the result
Overall, minor issues were identified and corrected.
Some pull-up resistors on the optical diodes were too large, preventing the communication to happen.
After the resistors were replaced, the optical communication was functional.

\subsubsection{Near-field current measurement}

The near-field current sensor is validated by injected a rectangular waveform on the sensed input.
On the output, a curve close the input waveform's derivative is obtained.
This result was expected because of the sensor's properties.
A much more detailed analysis in done in section \ref{sec:on-chip-near-field-process}.

\subsubsection{Testchip communication}

Finally, the integrated communication system is tested.
Unfortunaley, several issues were found with this system, completely preventing the communication.

% How is the test done
The test is conducted by sending the appropriate reading frame multiple times.
By design, the integrated communication system expects a \textit{clock} signal with a frequency lower than 2MHz, and an \textit{en\_in} signal that must be high for a single rising clock edge.
With both these criteria met, the communication system should return the data on the \textit{data\_out} pin.

% What is expected
By design, each frame incorporates a few mechanisms to ensure the returned data is correct.
Any \textit{data\_out} frame must start with binary code 1010 and end with binary code 01.
The \textit{enable} signal set on input \textit{en\_in} is propagated from one read cell to the next one.
The last cell of the chain is connected to the output pin \textit{en\_out}.
Confidence in the data is increased if the \textit{enable} signal (a pulse with a width of a single clock period) is observed on \textit{en\_out},

% What are the results
After attempting multiple readings with the same board, with different boards and different clock frequencies, the results are inconsistent.
Sometimes, the communication system returns an incomplete frame like in Fig. \ref{fig:read-only-partial-frame}.
In this case, the enable signal is not propagated correctly through the chain.
The \textit{en\_out} signal (not displayed here) stays low all the time.

%TODO: Add scale to label
\begin{figure}[!h]
  \centering
  \includegraphics[width=0.8\textwidth]{src/3/figures/read_only_partial_frame.pdf}
  \caption{Read-only partially returned frame (4ms/div, 5V/div) - clock frequency 1kHz}
  \label{fig:read-only-partial-frame}
\end{figure}

In other cases, the chain returns a complete but corrupted frame like in Fig. \ref{fig:read-only-full-frame}.
The enable is correctly propagated, and is visible on \textit{en\_out}, however the data start pattern is not correct (b'1011 instead of b'1010) and some intermediate digital values are not clearly defined.

%TODO: Add scale to label
\begin{figure}[!h]
  \centering
  \includegraphics[width=0.8\textwidth]{src/3/figures/read_only_corrupted_frame.pdf}
  \caption{Read-only corrupted frame (4ms/div, 5V/div) - clock frequency 1kHz}
  \label{fig:read-only-full-frame}
\end{figure}

%TODO: Detail more investigation
% Potential cause - clock freq
The clock frequency was initially suspected as a root cause for the problem.
Previous measurements were taken for a slow 1kHz clock frequency, which is far slower than the upper limit 2MHz.
Increasing the clock frequency at 10kHz, 100kHz and 1MHz did not improve the situation.

% Potential cause - delays
The impact of parasitic delays causing the communication system to malfunction was also suspected.
Multiple simulations of the entire read chain were run by placing large delays in multiple locations.
No failure could be highlighted using this method.

% Potential cause - parasitic RC
Afterwards, an RC parasitics were extracted from the layout.
Once again, the read chain performed correctly.

% Potential cause - susbstrate coupling
Susbtrate couplings were also suspected.
Some reading issues could be reproduced, but this type of coupling is highly unlikely to happen for clock frequencies of 1kHz.

\subsubsection{Future work}

% Multiple bonding diagrams
A second revision of the testchip is planned.
Since the investigation on the communication system is unconclusive, it was decided to remove it.
%TODO: Workaround ?
