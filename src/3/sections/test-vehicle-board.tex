\subsection{Test boards}

% Why these board
For providing the external devices required by the testchip, and communicating with it, a master-slave architecture is selected.
One \gls{pcb} is designed for the master, and one for the slave.
The slave board contains the testchip and the required external devices.
The master board contains mostly a microcontroller.
It is responsible for generating the frames for the testchip's monitoring system (section \ref{sec:comm-system-testchip}).
The master board connects to a computer for providing a user interface, reading, writing and storing the monitoring data.

% Overview of the setup
To protect the computer from the \gls{esd} discharges, the two boards are isolated electrically.
Each board is powered with its own isolated battery, to avoid conducted stress propagating in the AC supply network.
The communication between the two is achieved with several optical fibers.
This type of setup was successfully employed in [TODO: REF Alain] to measure \gls{spi} registers before and after electro-static discharges were applied on a product.

%TODO: System architecture
%TODO: Picture

\subsubsection{Master board}

% Architecture
The master board is actually just a shield for the microcontroller board Kinetis KL25Z.
It provides pin-compatible headers with the KL25Z, an isolated and regulated supply, and the devices required for optical communication.
The architecture is given in Fig. \ref{fig:master-board-architecture}

\begin{figure}[!h]
  \centering
  \includegraphics[width=0.3\textwidth]{src/3/figures/master_board_architecture.pdf}
  \caption{Master board architecture}
  \label{fig:master-board-architecture}
\end{figure}

%TODO: Picture

\subsubsection{Slave board}
The slave board provides two isolated and regulated supplies and devices required for the testchip and the optical communication.
The architecture is given in Fig. \ref{fig:slave-board-architecture}

\begin{figure}[!h]
  \centering
  \includegraphics[width=0.3\textwidth]{src/3/figures/slave_board_architecture.pdf}
  \caption{Slave board architecture}
  \label{fig:slave-board-architecture}
\end{figure}

%TODO: Picture

\subsubsection{Optical fiber communication}

% IO Spec - Pin connections
%TODO

% Opto-coupler references
To be done
